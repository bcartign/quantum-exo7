\documentclass[11pt,class=report,crop=false]{standalone}
\usepackage[screen]{../python}

%% Attention les figures sont commentées car longuettes à calculer !!
%\renewcommand{\commentfigure}[1]{\centerline{\emph{Ici une figure.}}}  % sans les figures compliquées
\renewcommand{\commentfigure}[1]{#1} % avec figures

\begin{document}


%====================================================================
\chapitre{Nombres complexes}
%====================================================================


\insertvideo{WMnoIdZwqWs}{partie 3.1. Écriture algébrique}

\insertvideo{oRmd3Y0l52U}{partie 3.2. Qubit}

\insertvideo{LvlqE14qJnw}{partie 3.3. Module - Argument}

\insertvideo{jChSroZEYPU}{partie 3.4. Écriture trigonométrique des qubits}

\insertvideo{EffE7nZvWwI}{partie 3.5. Sphère de Bloch}



\objectifs{Les nombres complexes sont les coefficients naturels des qubits. Nous détaillons les calculs avec les nombres complexes ainsi que sur les qubits.}



%%%%%%%%%%%%%%%%%%%%%%%%%%%%%%%%%%%%%%%%%%%%%%%%%%%%%%%%%%%%%%%%%%%%%
\section{Écriture algébrique}

Les nombres complexes étendent les nombres réels de façon à pouvoir résoudre les équations du type $x^2=-1$.


%--------------------------------------------------------------------
\subsection{Définition}

\begin{itemize}
  \item Un \defi{nombre complexe}\index{nombre complexe} est un couple $(a, b) \in \Rr^2$ que l'on notera $a + \ii b$. 
  
  \item Exemple avec $a=2$ et $b=3$ : $z = 2 + 3\ii$.  
  
  \item Le nombre complexe $\ii$ vérifie l'équation : \mybox{$\ii ^2 = - 1$}
  
  \myfigure{1}{
    \tikzinput{fig_complexes_01}
  } 
  
  \item \textbf{Addition.} $(a + \ii b) + (a' + \ii b') =
  (a + a') + \ii  (b + b')$
  \item \textbf{Multiplication} : $(a + \ii b) \times (a' + \ii b')
    = (aa' - bb') + \ii  (ab' + ba')$. 
    Ainsi on développe, en suivant les règles usuelles de la multiplication et
    en utilisant la règle $\ii ^2 = - 1$.
\end{itemize}

 
  
\begin{exemple}
Soit $z_1 = 2+3\ii$ et $z_2 = 5-4\ii$.

Alors
$$z_1+ z_2 = (2+3\ii) + (5-4\ii) = 7 - \ii. $$
  
Et
\begin{align*}
z_1  \times z_2
  &= (2+3\ii) \times (5-4\ii) \\
  &= 10 -8\ii +15\ii -12\ii^2 \\
  &= 10 -8\ii +15\ii +12 \\ 
  &= 22 +7\ii.
\end{align*}


\end{exemple}

%--------------------------------------------------------------
\subsection{Partie réelle et imaginaire}

Soit $z = a + \ii b$ un nombre complexe, sa \defi{partie réelle}\index{nombre complexe!partie reelle@partie réelle} est le réel $a$ et on
la note $\Re(z)$ ;
sa \defi{partie imaginaire}\index{nombre complexe!partie imaginaire} est le réel $b$ et on la note $\Im(z)$.



\myfigure{1}{
\tikzinput{fig_complexes_02}
}



%--------------------------------------------------------------------
\subsection{Module}



\textbf{Module.}
Le \defi{module}\index{nombre complexe!module} de $z = a + \ii b$ est le réel positif $|z| = \sqrt{a^2 + b^2}$. Il mesure la distance du point $(a,b)$ à l'origine $(0,0)$.
\myfigure{1}{
\tikzinput{fig_complexes_03}
}

Exemple : $|5-2\ii| = \sqrt{5^2+(-2)^2} = \sqrt{29}$.

\textbf{Nombres complexes de module $1$}. 
On peut représenter l'ensemble des nombres complexes de module $1$ par le cercle de rayon $1$ centré à l'origine.

\myfigure{1}{
\tikzinput{fig_complexes_05}
}

Exemples : $1$, $\ii$ et $\frac1{\sqrt2} + \frac1{\sqrt2}\ii$ sont des nombres complexes de module $1$.

On peut transformer un nombre complexe quelconque (non nul) en un nombre complexe de module $1$ en le divisant par son module. Par exemple $z = 5-2\ii$ a pour module $|z| = \sqrt{29}$, donc
$\frac{z}{|z|} = \frac5{\sqrt{29}} - \frac2{\sqrt{29}}\ii$ est de module $1$.


\textbf{Conjugué.}
Le \defi{conjugué}\index{nombre complexe!conjugue@conjugué} de $z = a + \ii b$ est $z^* = a - \ii b$, autrement dit
  $\Re(z^*) = \Re(z)$ et
  $\Im(z^*) = - \Im (z)$.
  Le point $z^*$ est le symétrique du point $z$ par rapport à l'axe réel.
Comme $z \times z^* = (a+\ii b)(a-\ii b) = a^2+b^2$ alors  le module vaut aussi $|z| = \sqrt{z z^*}$.
\myfigure{1}{
\tikzinput{fig_complexes_04}
}

\emph{Notation.} Une écriture plus classique pour le conjugué est $\bar z$, mais nous préférons ici la notation $z^*$ plus adaptée pour la suite du cours. 

\textbf{Inverse.}
L'\defi{\,inverse}: si $z \neq 0$, il existe un unique $z' \in \Cc$ tel
  que $zz' = 1$ (o\`u $1 = 1 + \ii  \times 0$).
\[ z' = \frac{1}{z}  =
     \frac{a - \ii b}{a^2 + b^2} = \frac{z^*}{|z|^2} . \]
%%%%%%%%%%%%%%%%%%%%%%%%%%%%%%%%%%%%%%%%%%%%%%%%%%%%%%%%%%%%%%%%%%%%%
\section{Qubit}

%--------------------------------------------------------------------
\subsection{Définition}

\index{qubit!definition@définition}

Rappelons la définition des qubits à partir des deux \defi{états quantiques de base} $\ket{0}$ et $\ket{1}$. 
Un \defi{$1$-qubit}, appelé aussi simplement \defi{qubit}, est un \defi{état quantique} obtenu par combinaison linéaire :
\mybox{$\ket{\psi} = \alpha \ket{0} + \beta \ket{1}
\qquad \text{ avec } \alpha \in \Cc \quad  \text{ et } \quad \beta \in \Cc$}
avec souvent la condition de normalisation :
$$|\alpha|^2+|\beta|^2 = 1.$$

Un qubit est donc défini par deux nombres complexes, $\alpha = a_1+\ii b_1$ et $\beta = a_2+\ii b_2$. Il faut ainsi $4$ nombres réels $a_1$, $b_1$, $a_2$, $b_2$ pour définir un qubit.


Deux qubits réunis sont dans un état quantique $\ket\psi$, appelé \defi{$2$-qubit}\index{qubit!deuxqubit@$2$-qubit}, défini par la superposition :
\mybox{$\ket\psi =
\alpha \ket{0.0} + \beta\ket{0.1} + \gamma\ket{1.0} + \delta \ket{1.1} \quad \text{ avec } \alpha,\beta,\gamma,\delta \in \Cc$}
avec souvent la convention de normalisation :
$$|\alpha|^2+|\beta|^2+|\gamma|^2+|\delta|^2=1.$$
Il faudrait donc $8$ nombres réels pour définir un $2$-qubit.

%--------------------------------------------------------------------
\subsection{Opérations}

\index{qubit!calculs}

\textbf{Addition.}
L'addition de deux qubits se fait coefficient par coefficient,  il s'agit donc d'additionner des paires de nombres complexes. Par exemple
si 
$$\ket\phi = (1+3\ii)\ket0 + 2\ii\ket1 \qquad \text{ et } \qquad
\ket\psi = 3\ket0 + (1-\ii) \ket1$$
alors
$$\ket\phi + \ket\psi  = (4+3\ii)\ket0 + (1+\ii)\ket1.$$
Ou encore pour des $2$-qubits :
$$\big(\ket{1.0}+\ket{0.1}\big) + \big(\ket{1.0}-\ket{0.1}\big) = 2\ket{1.0}.$$

\bigskip
\textbf{Multiplication.}
On peut multiplier deux $1$-qubits pour obtenir un $2$-qubit. Les calculs se font comme des calculs algébriques à l'aide des règles de bases $\ket0 \cdot \ket 0 = \ket{0.0}$, $\ket0 \cdot \ket 1 = \ket{0.1}$,\ldots{} 
Pour les coefficients, on utilise la multiplication des nombres complexes, avec bien sûr toujours la relation $\ii^2=-1$.

Par exemple avec $$\ket\phi = (1+3\ii)\ket0 + 2\ii\ket1 \qquad \text{ et } \qquad
\ket\psi = 3\ket0 + (1-\ii) \ket1$$
on a
\begin{align*}
\ket\phi \cdot \ket\psi
  &=\big((1+3\ii)\ket0 + 2\ii\ket1\big) \times \big(3\ket0 + (1-\ii) \ket1\big) \\
  &= (1+3\ii)\cdot3\cdot\ket0\cdot\ket0 
  + (1+3\ii)\cdot(1-\ii)\cdot\ket0\cdot\ket1+
    2\ii\cdot 3\cdot\ket1\cdot\ket0 + 2\ii\cdot(1-\ii)\cdot \ket1\cdot\ket1\\
  &= (3+9\ii)\ket{0.0}  + (4+2\ii)\ket{0.1} + 6\ii\ket{1.0}+(2+2\ii)\ket{1.1}
\end{align*}
où on a utilisé $(1+3\ii) \cdot (1-\ii) = 1-\ii+3\ii-3\ii^2 = 4+2\ii$ et $2\ii\cdot(1-\ii) = 2\ii-2\ii^2 = 2+2\ii$.

%--------------------------------------------------------------------
\subsection{Norme}

\index{qubit!norme}

\textbf{Norme.}
La norme d'un qubit est un nombre réel $\|\psi\|$.
\begin{itemize}
  \item Pour un qubit $\ket\psi = \alpha\ket0+\beta\ket1$, $\|\psi\| = \sqrt{|\alpha|^2+|\beta|^2}$ est sa norme.
  \item Pour un 2-qubit $\ket\psi = \alpha\ket{0.0}+\beta\ket{0.1}+\gamma\ket{1.0}+\delta\ket{1.1}$, sa norme est $\|\psi\| = \sqrt{|\alpha|^2+|\beta|^2+|\gamma|^2+|\delta|^2}$.
  \item La normalisation d'un qubit $\ket\psi$ est $\frac{\ket\psi}{\|\psi\|}$, qui  est un qubit de norme $1$.
\end{itemize}

Exemple : pour $\ket\psi = (3+4\ii)\ket0 + (2-\ii)\ket1$ alors la norme au carré vaut :
\begin{align*}
\| \psi \|^2 
  &= |3+4\ii|^2 + |2-\ii|^2 \\
  &= (3^2+4^2) + (2^2+(-1)^2) \\
  &= 30. 
\end{align*}
  Donc $\|\psi\| = \sqrt{30}$.

\begin{exercicecours}
Vérifier que la norme de 
$$\ket\psi = (1+\ii)\ket{0.0}  + (1-2\ii)\ket{0.1} + (3-4\ii)\ket{1.0}+2\ii\ket{1.1}$$
est $\|\psi\| = 6$.
Que vaut la normalisation de $\ket\psi$ ?
\end{exercicecours}


%%%%%%%%%%%%%%%%%%%%%%%%%%%%%%%%%%%%%%%%%%%%%%%%%%%%%%%%%%%%%%%%%%%%%
\section{Écriture trigonométrique}

%--------------------------------------------------------------------
\subsection{Module et argument}

Un nombre complexe $z\in\Cc$, admet l'écriture trigonométrique :
\mybox{
$z = r\cos\theta  + \ii r\sin\theta \qquad
\text{ avec } \quad r \in\Rr_+ \quad \text{ et } \quad \theta \in \Rr$}

\myfigure{1}{
\tikzinput{fig_complexes_06}
}

\begin{itemize}
  \item $r$ est en fait le module de $z$ : $r=|z|$,
  \item $\theta$ est un \defi{argument}\index{nombre complexe!argument} de $z$, on le note $\arg(z)$.
\end{itemize}

L'argument n'est pas unique : si $\theta$ est un argument alors $\theta + 2k\pi$ ($k\in\Zz$) aussi.
Pour rendre l'argument unique, on peut imposer la condition $\theta \in ]-\pi,\pi]$ (ou encore $\theta \in [0,2\pi[$). 
Si on impose $\theta \in ]-\pi,\pi]$ alors pour un nombre complexe $z$ non nul, l'écriture
$z = r\cos\theta  + \ii r\sin\theta$ est unique.

On dira que $\arg(z)$ est \og{}défini  modulo $2\pi$\fg{} et l'écriture 
$\theta \equiv \theta' \pmod{2\pi}$ signifie que $\theta = \theta'+2k\pi$ pour un certain entier $k\in\Zz$.
  
\begin{exemple}
\sauteligne
\begin{itemize}
  \item Soit $z = 1 - \sqrt3\ii$.
  Alors $r=|z| = 2$ et $\theta = -\frac\pi3$, car alors
  $$r\cos\theta = 2\cos(-\tfrac\pi3) = 2\times \frac12 = 1 = \Re(z)$$
  et 
  $$r\sin\theta = 2\sin(-\tfrac\pi3) = -2 \times \frac{\sqrt3}{2} = -\sqrt3 = \Im(z).$$  

\begin{center}
\begin{minipage}{0.45\textwidth}
\myfigure{0.7}{
\tikzinput{fig_complexes_07b}
}
\end{minipage}
\begin{minipage}{0.45\textwidth}
\myfigure{0.7}{
\tikzinput{fig_complexes_07a}
}
\end{minipage}  
\end{center}

  \item Le nombre complexe de module $r=3$ et d'argument $\theta=\frac{3\pi}{4}$
  est
  $$z' = r\cos\theta + \ii r\sin\theta = -\frac{3\sqrt2}{2} + \frac{3\sqrt2}{2}\ii
  = \frac{3\sqrt2}{2}(-1+\ii).$$
    
\end{itemize}
\end{exemple}  


L'écriture module-argument facilite le calcul des multiplications.
Les modules se multiplient, les arguments s'additionnent. 
\begin{proposition}
Soient $z$ et $z'$ deux nombres complexes.
Alors
\mybox{$|zz'| = |z| \cdot |z'| \qquad \text{ et } \qquad
\arg (zz') \equiv \arg (z) + \arg(z') \pmod {2\pi}$}
\end{proposition}
 
 
\begin{proof}
   \begin{eqnarray*}
     zz' & = & \left| z \right|  \left( \cos \theta + \ii  \sin \theta \right)
     \left| z' \right|  \left( \cos \theta' + \ii  \sin \theta' \right)\\
     & = & \left| zz' \right|  \left( \cos \theta \cos \theta' - \sin \theta
     \sin \theta' + \ii  \left( \cos \theta \sin \theta' + \sin \theta \cos
     \theta' \right) \right)\\
     & = & \left| zz' \right|  \left( \cos \left( \theta + \theta' \right) + \ii
     \sin \left( \theta + \theta' \right) \right)
   \end{eqnarray*}
   donc $|zz'| = |z| \cdot |z'|$ et $\arg( zz') \equiv \arg (z) + \arg(z') \pmod {2\pi}$.
 \end{proof} 
 

%--------------------------------------------------------------------
\subsection{Notation exponentielle}

Nous définissons la \defi{notation exponentielle}\index{nombre complexe!exponentielle} par
\mybox{$e^{\ii  \theta} = \cos \theta + \ii  \sin \theta$}
 et donc tout nombre complexe s'écrit :
\mybox{$z = r e^{\ii  \theta}$}
o\`u $r = \left| z \right|$ est son module et $\theta = \arg (z)$ est un de ses arguments.


Exemples : $e^{\ii\frac\pi2} = \ii$, $e^{\ii\pi} = -1$, $e^{2\ii \pi} = e^0 = 1$.


\bigskip

Avec la notation exponentielle, les calculs s'effectuent avec les lois habituelles pour les puissances. 
Par exemple :
\mybox{$\left(e^{\ii\theta}\right)^n = e^{\ii n \theta}$}
Il s'agit en fait de la \defi{formule de Moivre}\index{nombre complexe!formule de Moivre} qui s'écrit en version étendue :
$$\left( \cos \theta + \ii \sin \theta \right)^n = \cos \left( n \theta \right)
  + \ii  \sin \left( n \theta \right).$$
  
De façon plus générale, pour $z = r e^{\ii  \theta}$
et $z' = r' e^{\ii  \theta'}$, on peut écrire :
\begin{itemize}
	\item $zz' = r r' e^{\ii  \theta} e^{\ii  \theta'} = r r' e^{\ii  (\theta + \theta')}$
	\item $z^n = \left( r e^{\ii  \theta} \right)^n = r^n  \left( e^{\ii  \theta}
	\right)^n = r^n e^{\ii n \theta}$
	\item $1 / z = 1 / \left( r e^{\ii  \theta} \right) = \frac{1}{r} e^{- \ii
		\theta}$
	\item $z^* = r e^{-\ii \theta}$	
\end{itemize}

   
   
Tout nombre complexe de module $1$ s'écrit sous la forme $z=e^{\ii\theta}$,
autrement dit $z = \cos\theta +\ii\sin\theta$.
\myfigure{1}{
\tikzinput{fig_complexes_08}
}

%%%%%%%%%%%%%%%%%%%%%%%%%%%%%%%%%%%%%%%%%%%%%%%%%%%%%%%%%%%%%%%%%%%%%
\section{Écriture trigonométrique des qubits}


%--------------------------------------------------------------------
\subsection{Écriture des qubits}

À l'aide de la notation exponentielle, un qubit $\ket\psi = \alpha \ket0 + \beta \ket1$ peut aussi s'écrire :
$$\ket \psi = r e^{\ii\theta} \ket0 + r'e^{\ii\theta'} \ket1.$$
Un tel qubit est normalisé si $r^2+r'^2=1$.

Certains utilisent un vocabulaire issu de la physique :
\begin{itemize}
  \item $\theta$ est la \defi{phase} associée à $\ket0$,
  \item $\theta'$ est la phase associée à $\ket1$.
\end{itemize}

L'écriture algébrique est adaptée à un calcul de somme tandis que la notation exponentielle rend le calcul d'une multiplication plus facile.


\begin{exemple}
Si $\ket\phi = 2e^{\ii \frac\pi3}\ket0+3e^{\ii\frac\pi4}$ et
$\ket\psi = 4e^{\ii \frac\pi5}\ket0+5e^{\ii\frac\pi6}$.
Alors :
\begin{align*}
\ket\phi \cdot \ket\psi 
  &= \left(2e^{\ii \frac\pi3}\ket0+3e^{\ii\frac\pi4}\ket1\right) \times
  \left(4e^{\ii \frac\pi5}\ket0+5e^{\ii\frac\pi6}\ket1\right) \\
  &= 2e^{\ii \frac\pi3} \cdot 4e^{\ii \frac\pi5}\ket{0.0}
  + 2e^{\ii \frac\pi3} \cdot 5e^{\ii\frac\pi6} \ket{0.1}
  + 3e^{\ii\frac\pi4} \cdot 4e^{\ii \frac\pi5}\ket{1.0} 
  + 3e^{\ii\frac\pi4} \cdot 5e^{\ii\frac\pi6} \ket{1.1} \\
  &= 8e^{\ii (\frac\pi3+\frac\pi5)}\ket{0.0}
    + 10e^{\ii (\frac\pi3+\frac\pi6)} \ket{0.1}
    + 12e^{\ii(\frac\pi4+\frac\pi5)}\ket{1.0} 
    + 15e^{\ii(\frac\pi4+\frac\pi6)} \ket{1.1}. \\
\end{align*}
\end{exemple}


%--------------------------------------------------------------------
\subsection{Équivalence de qubits}

La mesure physique d'un qubit $\ket\psi = \alpha\ket0+\beta\ket1$ ne permet pas d'accéder aux valeurs de $\alpha$ et $\beta$.
Par exemple $\ket0+\ket1$ et $2\ket0+2\ket1$ ne pourront pas être distingués par des mesures, ils donnent tous les deux une mesure $0$ ou $1$ avec probabilité $1/2$.



On dit que deux états sont \defi{équivalents}\index{qubit!equivalence@équivalence} si on peut passer de l'un à l'autre par les opérations suivantes :
\begin{itemize}
  \item multiplication par une constante réelle :
$$k \ket\psi \equiv \ket \psi \qquad \text{ où } k \in \Rr^*,$$
  \item multiplication par $e^{\ii\theta}$ (un nombre complexe de module $1$) :
$$e^{\ii \theta} \ket\psi \equiv \ket \psi \qquad \text{ où } \theta \in \Rr.$$
\end{itemize}

Une reformulation est de dire que deux qubit $\ket\phi$ et $\ket\psi$ sont \defi{équivalents} s'il existe $z\in\Cc^*$ tel que $\ket\phi = z \ket\psi$.

Deux états quantiques équivalents ne peuvent pas être distingués par des mesures.

\begin{exemple}
\sauteligne
\begin{itemize}
  \item Par exemple 
$$\ket0 + \ket1 \equiv \frac{1}{\sqrt2}\ket0+\frac{1}{\sqrt2}\ket1.$$
On passe de l'un à l'autre en multipliant par $k=1/\sqrt2$.

  \item On a aussi
$$\ii\ket0 + (1-\ii) \ket1 \equiv -\ket 0 + (1+\ii)\ket1$$
On passe de l'un à l'autre en multipliant par $\ii = e^{\ii \frac\pi2}$.
 
  \item On peut combiner les deux opérations :
$$(1+2\ii)\ket0 + \ii \ket1 \equiv \ket 0 + \frac{2+\ii}{5}\ket1$$
On passe de l'un à l'autre en multipliant par $z=\frac{1-2\ii}{5}$.
Les deux qubits équivalents $(1+2\ii)\ket0 + \ii \ket1$ et $\ket 0 + \frac{2+\ii}{5}\ket1$ conduisent tous les deux lors d'une mesure à $0$ avec une probabilité $\frac{5}{6}$ et à $1$ avec une probabilité $\frac{1}{6}$.
\end{itemize}
\end{exemple}


\begin{remarque*}
\sauteligne
\begin{itemize}
  \item Attention, deux états équivalents ne sont pas égaux ! Il ne faut pas les interchanger dans les calculs intermédiaires. Cependant, lors de la mesure finale, on peut remplacer un état par un état équivalent sans changer le résultat. 
  \item En effet, les deux opérations élémentaires qui définissent l'équivalence ne changent pas le calcul de probabilité pour la mesure.
  \item L'équivalence des qubits évite en particulier de parler de normalisation.
\end{itemize}
\end{remarque*}


\begin{proposition}
\label{prop:dimdeux}
Un qubit $\ket\psi = \alpha\ket0+\beta\ket1$ non nul (c'est-à-dire $(\alpha,\beta)\neq(0,0)$) est équivalent à un qubit de norme $1$ de la forme :
\mybox{$\ket{\psi} \equiv \cos\left(\frac\theta2\right)\ket0+ \sin\left(\frac\theta2\right)e^{\ii\phi}\ket1$}
De plus l'écriture est unique lorsque l'on a les conditions $0<\theta<\pi$ et $-\pi < \phi \le \pi$.
\end{proposition}


\begin{exemple}
\sauteligne
\begin{itemize}
  \item $\ket\psi = \ii\sqrt2\ket0 + \sqrt3(1+\ii)\ket1$.
  On commence par rendre le coefficient devant $\ket0$ réel positif. Pour cela on multiplie tout par $-\ii = e^{-\ii\frac\pi2}$ :
  $$\ket\psi \equiv (-\ii)\big(\ii\sqrt2\ket0 + \sqrt3(1+\ii)\ket1\big)
  = \sqrt2\ket0 + \sqrt3(1-\ii)\ket1.$$
  La norme de ce dernier qubit est $2\sqrt2$, on divise donc par cette norme. Ainsi :
  $$\ket\psi \equiv \frac12\ket0 + \frac{\sqrt3}{2} \frac{1-\ii}{\sqrt2}\ket1.$$
  On pose d'une part $\theta=\frac{2\pi}3$ pour lequel $\cos\left(\frac\theta2\right)=\frac12$ et $\sin\left(\frac\theta2\right)= \frac{\sqrt3}{2}$ et d'autre part $\phi=-\frac{\pi}{4}$ pour lequel
  $e^{\ii\phi} = \frac{1-\ii}{\sqrt2}$.
  
 
  \item Il n'est en général pas possible d'expliciter les angles $\theta$ et $\phi$.
  Considérons par exemple $\ket\psi = 1\ket0 - 2\ii\ket1$.
  Alors $\|\psi\| = \sqrt5$ et
  $\ket\psi \equiv \frac{1}{\sqrt5}\ket0 -\frac{2}{\sqrt5}\ii\ket1$.
  On sait qu'il existe un réel $\theta$ tel que $\cos\left(\frac\theta2\right) = \frac{1}{\sqrt5}$
  et $\sin\left(\frac\theta2\right) = \frac{2}{\sqrt5}$, ce $\theta$ est défini par 
  $\frac\theta2 = \operatorname{arccos}(\frac{1}{\sqrt5})$ mais n'a pas d'expression plus explicite.
  
  On pose alors $\phi = -\frac\pi2$ qui vérifie $e^{\ii\phi} = -\ii$.
  Ces $\theta$ et $\phi$ conviennent.
  
\end{itemize} 
\end{exemple}

\begin{proof}
~

\textbf{Existence.}
On commence par transformer le coefficient de $\ket0$ en un réel positif. Si $\alpha = re^{\ii\theta}$ alors
\begin{align*}
\ket\psi 
  &\equiv e^{-\ii\theta}\ket\psi \\
  &= e^{-\ii\theta}\big( re^{\ii\theta} \ket0 + \beta\ket1\big) \\
  &= r\ket0 + \beta \cdot e^{-\ii\theta} \ket1. \\
\end{align*}
On normalise ensuite ce qubit en divisant par $\|\psi\|$ :
\begin{align*}
\ket\psi 
  &\equiv \frac{1}{\|\psi\|} \ket\psi \\
  &\equiv \frac{1}{\|\psi\|}\big(r\ket0 + \beta \cdot e^{-\ii\theta} \ket1\big) \\
  &= \frac{r}{\|\psi\|}\ket0 + \frac{\beta \cdot e^{-\ii\theta}}{\|\psi\|} \ket1. \\
\end{align*}
Ce dernier qubit s'écrit :
$$\ket{\psi'} = r'\ket0 + \beta'\ket1$$
avec $r' \in \Rr$ et $\beta'\in \Cc$. 
Mais comme par définition $\ket{\psi'}$ est un qubit de module $1$, on a de plus 
$r'^2+ |\beta'|^2 = 1$
et en particulier $0\le r' \le 1$ et $|\beta'| \le 1$. 

\bigskip
\emph{Rappel.} Pour deux nombres réels $a,b\ge0$ vérifiant $a^2+b^2=1$, il existe $x\in[0,\frac\pi2]$ tel que
$$\left\{\begin{array}{rcl}
a &=& \cos x\\
b &=& \sin x\\
\end{array}\right.$$

\myfigure{1}{
\tikzinput{fig_complexes_09}
}

On applique le rappel à $r'$ et $|\beta'|$ afin d'obtenir $x = \frac\theta2$ ($\theta \in [0,\pi]$), et d'écrire
$r' = \cos\left(\frac\theta2\right)$ et $|\beta'| = \sin\left(\frac\theta2\right)$.
Finalement, $\beta' = \sin\left(\frac\theta2\right) e^{\ii\phi}$ pour un certain argument $\phi\in\Rr$.
Ainsi $\ket\psi$ est bien équivalent à un qubit de la forme souhaitée : 
$$\ket\psi \equiv \ket{\psi'} = \cos\left(\frac\theta2\right)\ket0+ \sin\left(\frac\theta2\right)e^{\ii\phi}\ket1.$$

\bigskip
On aurait pu effectuer toutes les opérations en une seule fois. En effet,
si $\ket\psi = \alpha\ket0 + \beta\ket1$ avec $\alpha = re^{\ii\theta}$ alors
$\frac{e^{-\ii\theta}}{\|\psi\|} \ket\psi$ est de la forme voulue.

\bigskip
\textbf{Unicité.}
L'unicité découle de la construction. On peut aussi la prouver de la façon suivante. Si on suppose qu'il existe $(\theta,\phi)$ et $(\theta',\phi')$ tels que
$$\cos\left(\frac\theta2\right)\ket0+ \sin\left(\frac\theta2\right)e^{\ii\phi}\ket1 = \cos\left(\frac{\theta'}2\right)\ket0+ \sin\left(\frac{\theta'}2\right)e^{\ii\phi'}\ket1$$
alors, par identification, les coefficients devant $\ket0$ sont égaux, de même pour les coefficients devant $\ket1$.
Ainsi
$$\left\{\begin{array}{rcl}
\cos(\frac\theta2) &=& \cos(\frac{\theta'}2) \\
\sin(\frac{\theta}2)e^{\ii\phi} &=& \sin(\frac{\theta'}2)e^{\ii\phi'}
\end{array}\right.$$

Mais comme $0 <\frac\theta2< \frac\pi2$ et $0 <\frac{\theta'}{2} < \frac\pi2$ alors
$\cos(\frac\theta2) = \cos(\frac{\theta'}2)$ implique $\theta=\theta'$.
On en déduit donc que $\sin(\frac\theta2) = \sin(\frac{\theta'}2)$,
puis que $e^{\ii\phi} = e^{\ii\phi'}$. Deux arguments sont égaux modulo $2\pi$, mais comme on a imposé 
$-\pi < \phi \le \pi$ et $-\pi < \phi' \le \pi$, on a $\phi=\phi'$.
\end{proof}



%%%%%%%%%%%%%%%%%%%%%%%%%%%%%%%%%%%%%%%%%%%%%%%%%%%%%%%%%%%%%%%%%%%%%
\section{Sphère de Bloch}

\index{sphere de Bloch@sphère de Bloch}

%--------------------------------------------------------------------
\subsection{Représentation}

Un qubit $\ket\psi = \alpha\ket0+\beta\ket1$ est déterminé par ses $2$ coefficients complexes $\alpha,\beta$, donc 
par $4$ paramètres réels $\Re(\alpha)$, $\Im(\alpha)$, $\Re(\beta)$, $\Im(\beta)$.

Mais un qubit est équivalent à un qubit de la forme :
$$\ket{\psi'} = \cos\left(\frac\theta2\right)\ket0+ \sin\left(\frac\theta2\right)e^{\ii\phi}\ket1$$
avec seulement deux paramètres réels $\theta$, $\phi$
qui vérifient $0\le \theta \le \pi$ et $-\pi < \phi \le \pi$.

Cela permet de représenter un qubit sur la \defi{sphère de Bloch},
par un point (ou un vecteur) de colatitude $\theta$ et longitude $\phi$, et un rayon $1$.
\commentfigure{
\myfigure{1}{
\tikzinput{bloch01}
}
}

Les formules pour obtenir les coordonnées $(x,y,z)\in\Rr^3$ de ce point sont :
$$\left\{\begin{array}{rcl}
x &=& \sin\theta \cdot \cos\phi \\
y &=& \sin\theta \cdot \sin\phi \\
z &=& \cos\theta \\
\end{array}\right.$$


\bigskip
\textbf{États quantiques de base.}
L'état de base $\ket0$ correspond au pôle Nord de coordonnées $(x,y,z)= (0,0,1)$ avec pour colatitude $\theta=0$ (et n'importe quelle valeur comme longitude $\phi$).
 \commentfigure{ 
\myfigure{0.8}{
\tikzinput{bloch02a}\qquad\qquad
\tikzinput{bloch02b}
}  
}
L'état de base $\ket1$ correspond au pôle Sud de coordonnées $(x,y,z)= (0,0,-1)$ avec pour colatitude $\theta=\pi$ (ou $180^\circ$) (et n'importe quelle valeur comme longitude $\phi$).
 
\bigskip
\textbf{Mesure.}
Pour un qubit $\ket\psi = \cos\left(\frac\theta2\right)\ket0+ \sin\left(\frac\theta2\right)e^{\ii\phi}\ket1$, la probabilité que sa mesure donne $0$ est
$\cos^2(\frac\theta2)$. 
Ainsi, si le qubit est plus proche du pôle Nord, c'est-à-dire $0 \le \theta < \frac\pi2$, alors la probabilité de mesurer $0$ est plus forte.
Par contre, si le qubit est plus proche du pôle Sud, c'est-à-dire $\frac\pi2 < \theta \le  \pi$, alors la probabilité de mesurer $1$ est plus forte.
Un qubit sur l'équateur se mesure en $0$ ou $1$ avec la même probabilité.

\myfigure{1}{
\tikzinput{bloch-proba}
}


\begin{exemple}
L'état $\ket{\psi_+} = \frac1{\sqrt2} \big(\ket0+\ket1\big)$ a pour paramètres
$\theta = \frac\pi2$ (ou $90^\circ$) et $\phi=0$. 
  
  
  L'état $\ket{\psi_-} = \frac1{\sqrt2} \big(\ket0-\ket1\big)$ a pour paramètres
   $\theta = \frac\pi2$ et $\phi=\pi$.
    
   \commentfigure{
   \myfigure{1}{
   \tikzinput{bloch03}
   }  
   }
Ils sont tous les deux situés sur l'équateur, donc se mesurent en $0$ ou $1$ avec une probabilité $\frac12$.
\end{exemple}


\emph{Remarque.} On rappelle que l'écriture d'un qubit sous la forme $ \cos\left(\frac\theta2\right)\ket0+ \sin\left(\frac\theta2\right)e^{\ii\phi}\ket1$ ne s'obtient qu'à équivalence près. Aussi la représentation sur la sphère de Bloch n'est que partielle et ne permet pas une représentation complète d'un qubit.


\begin{exercicecours}
\sauteligne
\begin{enumerate}
  \item Tracer les points suivant sur la sphère de Bloch :
  \begin{enumerate}
    \item Les points de coordonnées sphériques $(\theta,\phi)=(\frac\pi3,-\frac\pi4)$ et $(\theta,\phi)=(\frac\pi2,\frac{2\pi}{3})$.
    Exprimer les qubits correspondants. 
    
    \item Les points de coordonnées cartésiennes $(x,y,z) = (0,-1,0)$
    et $(\frac{1}{\sqrt2},-\frac{1}{\sqrt2},0)$, $(\frac{1}{\sqrt2},0,-\frac1{\sqrt2})$. Exprimer les qubits correspondants. 
    
    \item Les points des états $\ket{\psi_1} = -\frac12\ket0 - \frac{\sqrt3}{2}e^{\ii\pi}\ket1$ (attention au facteur $2$ dans la formule $\frac\theta2$ !) et    
    $\ket{\psi_2} = \ket0 + \ii\ket1$ (penser à normaliser).
    
  \end{enumerate}
  
  \item Trouver les valeurs des qubits suivants placés sur la sphère de Bloch.
  Une lecture graphique approximative des valeurs $\theta$ et $\phi$ suffit.
  
  \commentfigure{
  \myfigure{0.8}{
  \tikzinput{bloch05a}\qquad\qquad
  \tikzinput{bloch05b}
  } 
  }

  \item Trouver l'expression des états quantiques situés sur l'équateur.
  
  
  \item À quelle transformation géométrique correspond la transformation d'un qubit $\ket\psi = \cos\left(\frac\theta2\right)\ket0+ \sin\left(\frac\theta2\right)e^{\ii\phi}\ket1$
  en $\ket{\psi'} = \cos\left(\frac\pi2-\frac\theta2\right)\ket0+ \sin\left(\frac\pi2-\frac\theta2\right)e^{\ii\phi}\ket1$ ?
  
  \item Trouver l'expression de la symétrie centrale de centre l'origine.
  Exprimez d'abord la transformation $(\theta,\phi) \mapsto (\theta',\phi')$
  puis l'action $\ket\psi \mapsto \ket{\psi'}$.
  
\end{enumerate}
\end{exercicecours}


%--------------------------------------------------------------------
\subsection{Portes X, Y et Z de Pauli}

\index{porte!X}
\index{porte!Y}
\index{porte!Z}
\index{porte!de Pauli}

Les portes de Pauli $X$, $Y$ et $Z$ transforment un qubit en un autre qubit.
Elles ont chacune une interprétation géométrique simple lorsque l'on regarde leur action sur la sphère de Bloch.

\textbf{Porte $X$.}
La porte $X$ échange les coefficients d'un qubit :
$$
\text{ porte } X \ : \ 
\left\lbrace
\begin{array}{l}
\ket 0 \mapsto \ket1 \\
\ket 1 \mapsto \ket0
\end{array}\right.
$$
Autrement dit :
$$X(\alpha\ket0+\beta\ket1) = \beta\ket0+\alpha\ket1.$$


Regardons ce que cela donne sur la sphère de Bloch.
Soit $\ket\psi = \cos\left(\frac\theta2\right) \ket0 + \sin\left(\frac\theta2\right)e^{\ii\phi}\ket1$, alors
\begin{align*}
X(\ket\psi) 
  &= \sin\tfrac\theta2e^{\ii\phi}\ket0 + \cos\tfrac\theta2 \ket1 \\
  &\equiv \sin\tfrac\theta2 \ket0 +  \cos\tfrac\theta2 e^{-\ii\phi}\ket1 \\
  &= \cos\left(\tfrac\pi2-\tfrac\theta2\right)\ket0 + \sin\left(\tfrac\pi2-\tfrac\theta2\right)e^{-\ii\phi}\ket1 \\
  &=  \cos\tfrac{\theta'}2 \ket0 + \sin\tfrac{\theta'}2e^{\ii\phi'}\ket1 \quad \text{ avec } \theta' = \pi-\theta \text{ et } \phi'\equiv-\phi \pmod{2\pi}\\
\end{align*}

Ainsi les coordonnées sphériques $(\theta,\phi)$ sont transformées par $X$ en $(\pi-\theta,-\phi)$.
Géométriquement $X(\ket\psi)$ est obtenu sur la sphère de Bloch par la rotation (de l'espace) d'axe $(Ox)$ et d'angle $\pi$ (un demi-tour donc).
\commentfigure{
\myfigure{1}{
\tikzinput{bloch04a}
} 
}


\textbf{Portes $Y$ et $Z$.}
Rappelons l'action des portes de Pauli $Y$ et $Z$ sur les états de base :
$$\text{ porte } Y \ : \ \left\lbrace
\begin{array}{l}
\ket 0 \mapsto \ii\ket 1 \\
\ket 1 \mapsto -\ii\ket 0
\end{array}\right.
\qquad\qquad
\text{ porte } Z  \ : \ 
\left\lbrace
\begin{array}{l}
\ket 0 \mapsto \ket0 \\
\ket 1 \mapsto -\ket1
\end{array}\right.$$

Géométriquement la porte $Y$ correspond à la rotation d'axe $(Oy)$ et d'angle $\pi$.
Les coordonnées $(\theta,\phi)$ sont transformées en $(\pi-\theta,\pi-\phi)$.


La porte $Z$ correspond à la rotation d'axe $(Oz)$ et d'angle $\pi$.
Les coordonnées $(\theta,\phi)$ sont transformées en $(\theta,\phi+\pi)$.

\commentfigure{
\myfigure{0.8}{
\tikzinput{bloch04b}\qquad\qquad
\tikzinput{bloch04c}
} 
}

\end{document}
