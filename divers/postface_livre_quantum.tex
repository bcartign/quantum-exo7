
\clearemptydoublepage
\pagestyle{empty}\thispagestyle{empty}

\vspace*{\fill}

\section*{Notes et bibliographie}

Le livre de référence est \emph{Quantum computation and quantum information} (Cambridge university press, 2010) de Mickael Nielsen et Isaac Chuang. Cet ouvrage est à la fois abordable et complet.

\medskip

La première partie de ce cours est inspirée d'une série de vidéos de Mickael Nielsen :
\mycenterline{
\href{https://www.youtube.com/watch?v=X2q1PuI2RFI&list=PL1826E60FD05B44E4}
{\og{}\emph{Quantum computing for the determined}\fg{}}
}

\medskip

Un des rares cours en français est \og{}Introduction à l'informatique quantique\fg{} par Y.~Leroyer et G.~Sénizergues
à l'Enseirb-Matmeca \href{https://dept-info.labri.fr/~ges/ENSEIGNEMENT/CALCULQ/polycop_calculq.pdf}{disponible ici}.

\medskip

Le chapitre \og{}Algorithme de Shor\fg{} est basé sur l'article \emph{Shor's algorithm for factoring large integers} par C.~Lavor, L.R.U.~Manssur, R.~Portugal (2003), disponible \href{https://arxiv.org/abs/quant-ph/0303175}{sur arXiv}.


\section*{Remerciements}

Je remercie Michel Bodin, Stéphanie Bodin, François Recher et Jean-Michel Torres pour leurs relectures.






\bigskip

\begin{center}
Vous pouvez récupérer l'intégralité des codes \Python{} ainsi que tous les fichiers sources sur la page \emph{GitHub} d'Exo7 :
\href{https://github.com/exo7math/quantum-exo7}{\og{}GitHub : Quantum\fg{}}.



\end{center}


\vspace*{\fill}

\bigskip 

\begin{center}
\LogoExoSept{3}
\end{center}



\begin{center}
Ce livre est diffusé sous la licence \emph{Creative Commons -- BY-NC-SA -- 4.0 FR}.

Sur le site Exo7 vous pouvez télécharger gratuitement le livre en couleur.
\end{center}




\pagenumbering{gobble} % remove page numbering 
\printindex
\pagenumbering{arabic}

